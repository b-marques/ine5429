%% abtex2-modelo-artigo.tex, v-1.9.6 laurocesar
%% Copyright 2012-2016 by abnTeX2 group at http://www.abntex.net.br/ 
%%
%% This work may be distributed and/or modified under the
%% conditions of the LaTeX Project Public License, either version 1.3
%% of this license or (at your option) any later version.
%% The latest version of this license is in
%%   http://www.latex-project.org/lppl.txt
%% and version 1.3 or later is part of all distributions of LaTeX
%% version 2005/12/01 or later.
%%
%% This work has the LPPL maintenance status `maintained'.
%% 
%% The Current Maintainer of this work is the abnTeX2 team, led
%% by Lauro César Araujo. Further information are available on 
%% http://www.abntex.net.br/
%%
%% This work consists of the files abntex2-modelo-artigo.tex and
%% abntex2-modelo-references.bib
%%

% ------------------------------------------------------------------------
% ------------------------------------------------------------------------
% abnTeX2: Modelo de Artigo Acadêmico em conformidade com
% ABNT NBR 6022:2003: Informação e documentação - Artigo em publicação 
% periódica científica impressa - Apresentação
% ------------------------------------------------------------------------
% ------------------------------------------------------------------------

\documentclass[
    % -- opções da classe memoir --
    article,            % indica que é um artigo acadêmico
    11pt,               % tamanho da fonte
    oneside,            % para impressão apenas no recto. Oposto a twoside
    a4paper,            % tamanho do papel. 
    % -- opções da classe abntex2 --
    %chapter=TITLE,     % títulos de capítulos convertidos em letras maiúsculas
    %section=TITLE,     % títulos de seções convertidos em letras maiúsculas
    %subsection=TITLE,  % títulos de subseções convertidos em letras maiúsculas
    %subsubsection=TITLE % títulos de subsubseções convertidos em letras maiúsculas
    % -- opções do pacote babel --
    english,            % idioma adicional para hifenização
    brazil,             % o último idioma é o principal do documento
    sumario=tradicional,
    ]{abntex2}


% ---
% PACOTES
% ---
\usepackage[table,xcdraw]{xcolor}

% ---
% Pacotes fundamentais 
% ---
\usepackage{lmodern}            % Usa a fonte Latin Modern
\usepackage[T1]{fontenc}        % Selecao de codigos de fonte.
\usepackage[utf8]{inputenc}     % Codificacao do documento (conversão automática dos acentos)
% \usepackage{indentfirst}        % Indenta o primeiro parágrafo de cada seção.
\usepackage{nomencl}            % Lista de simbolos
\usepackage{color}              % Controle das cores
\usepackage{graphicx}           % Inclusão de gráficos
\usepackage{microtype}          % para melhorias de justificação
% ---
        
% ---
% Pacotes adicionais, usados apenas no âmbito do Modelo Canônico do abnteX2
% ---
\usepackage{lipsum}             % para geração de dummy text
\usepackage{fancyvrb}
\usepackage{todonotes}
\usepackage{float}
\usepackage[procnames]{listings}
\usepackage{color}
\usepackage{amsmath}
% ---
        
% ---
% Pacotes de citações
% ---
\usepackage[brazilian,hyperpageref]{backref}     % Paginas com as citações na bibl
\usepackage[alf]{abntex2cite}   % Citações padrão ABNT
% ---

% ---
% Configurações do pacote backref
% Usado sem a opção hyperpageref de backref
\renewcommand{\backrefpagesname}{Citado na(s) página(s):~}
% Texto padrão antes do número das páginas
\renewcommand{\backref}{}
% Define os textos da citação
\renewcommand*{\backrefalt}[4]{
    \ifcase #1 %
        Nenhuma citação no texto.%
    \or
        Citado na página #2.%
    \else
        Citado #1 vezes nas páginas #2.%
    \fi}%
% ---

% ---
% Informações de dados para CAPA e FOLHA DE ROSTO
% ---
\titulo{Relatório INE5429 - Segurança em Computação\\ 
        Pseudo-Aleatórios}
\autor{Bruno Marques do Nascimento\thanks{brunomn95@gmail.com \hspace{1mm} - \hspace{1mm} Universidade Federal de Santa Catarina}}
\instituicao{Universidade Federal de Santa Catarina}
\local{Florianópolis - SC, Brasil}
\data{14 de Abril de 2018}
% ---

% ---
% Configurações de aparência do PDF final

% alterando o aspecto da cor azul
\definecolor{blue}{RGB}{41,5,195}

% informações do PDF
\makeatletter
\hypersetup{
        %pagebackref=true,
        pdftitle={\@title}, 
        pdfauthor={\@author},
        pdfsubject={Modelo de artigo científico com abnTeX2},
        pdfcreator={LaTeX with abnTeX2},
        pdfkeywords={abnt}{latex}{abntex}{abntex2}{atigo científico}, 
        colorlinks=true,            % false: boxed links; true: colored links
        linkcolor=blue,             % color of internal links
        citecolor=blue,             % color of links to bibliography
        filecolor=magenta,              % color of file links
        urlcolor=blue,
        bookmarksdepth=4
}
\makeatother
% --- 

% ---
% compila o indice
% ---
\makeindex
% ---

% ---
% Altera as margens padrões
% ---
\setlrmarginsandblock{3cm}{3cm}{*}
\setulmarginsandblock{3cm}{3cm}{*}
\checkandfixthelayout
% ---

% --- 
% Espaçamentos entre linhas e parágrafos 
% --- 

% O tamanho do parágrafo é dado por:
\setlength{\parindent}{1.3cm}

% Controle do espaçamento entre um parágrafo e outro:
\setlength{\parskip}{0.2cm}  % tente também \onelineskip

% Espaçamento simples
\SingleSpacing

% ----
% Início do documento
% ----
\begin{document}

% Seleciona o idioma do documento (conforme pacotes do babel)
%\selectlanguage{english}
\selectlanguage{brazil}

% Retira espaço extra obsoleto entre as frases.
\frenchspacing 

% ----------------------------------------------------------
% ELEMENTOS PRÉ-TEXTUAIS
% ----------------------------------------------------------

%---
%
% Se desejar escrever o artigo em duas colunas, descomente a linha abaixo
% e a linha com o texto ``FIM DE ARTIGO EM DUAS COLUNAS''.
% \twocolumn[           % INICIO DE ARTIGO EM DUAS COLUNAS
%
%---
% página de titulo

\maketitle


% ----------------------------------------------------------
% ELEMENTOS TEXTUAIS
% ----------------------------------------------------------
\textual

% ----------------------------------------------------------
% Introdução
% ----------------------------------------------------------
% \section*{Introdução}
% \addcontentsline{toc}{section}{Introdução}

\section*{\textbf{Algoritmos:}}
\addcontentsline{toc}{section}{Questões e respostas:}

% ----------------------------------------------------------
% Questão 1
% ----------------------------------------------------------
\subsection*{\textbf{1 - Blum Blum Shub(BBS):}}
\addcontentsline{toc}{subsection}{1 - Blum Blum Shub}

O Blum Blum Shub é um algoritmo que se baseia na seguinte fórmula: $x_i = x_{i-1}^2 \mod n$, no qual a saída é a concatenação dos bits de paridade ($z_1,z_2,z_3...$), que valem 0 quando $x_i$ é par e 1 quando $x_i$ é ímpar. Para este algoritmo são necessário dois números primos grandes ($p$ e $q$) ambos congruentes a $3\pmod4$, também uma semente aleatória $s$ $\in$ $[1,1-n]$, e $x_0 = s^2\mod n$, onde $n = p * q$.

\begin{table}[H]
\centering
\caption{Blum Blum Shub}
\label{bbs-table}
\begin{tabular}{|c|c|}
\hline
\rowcolor[HTML]{C0C0C0} 
\textbf{Tamanho do Número} & \textbf{\begin{tabular}[c]{@{}c@{}}Tempo para gerar\\ (microssegundos)\end{tabular}} \\ \hline
40                         & 29.80                                                                                \\ \hline
56                         & 60.08                                                                                \\ \hline
80                         & 75.82                                                                                \\ \hline
128                        & 87.02                                                                                \\ \hline
168                        & 153.54                                                                               \\ \hline
224                        & 230.79                                                                               \\ \hline
256                        & 183.11                                                                               \\ \hline
512                        & 387.19                                                                               \\ \hline
1024                       & 875.95                                                                               \\ \hline
2048                       & 1487.02                                                                              \\ \hline
4096                       & 3154.28                                                                              \\ \hline
\end{tabular}
\end{table}


\definecolor{keywords}{RGB}{255,0,90}
\definecolor{comments}{RGB}{0,0,113}
\definecolor{red}{RGB}{160,0,0}
\definecolor{green}{RGB}{0,150,0}
\lstset{language=Python, 
        basicstyle=\ttfamily\small, 
        keywordstyle=\color{keywords},
        commentstyle=\color{comments},
        stringstyle=\color{red},
        showstringspaces=false,
        identifierstyle=\color{green},
        procnamekeys={def,class},
        inputencoding=utf8,
        literate={á}{{\'a}}1 {ó}{{\'o}}1 {é}{{\'e}}1 {í}{{\'i}}1 {ú}{{\'u}}1
             {ã}{{\~a}}1 {õ}{{\~o}}1
             {ç}{{\c{c}}}1,
}

\begin{lstlisting}[frame=single, title={blum-blum-shub.py}]
import sys
import time as t
MICRO = 1000*1000

p = 3141592653589771  # Número primo p congruente a 3 (mod 4).
q = 2718281828459051  # Número primo q congruente a 3 (mod 4).
seed = 2              # Semente aleatória entre [1, n-1].
random = 0            # Número aleatório que será gerado.
x = []                # Array dos valores de x.

n = p * q

bits = int(sys.argv[1]) + 1

start = t.time() # Tempo de início da execução.

x.append(seed*seed % n) # x[0] := seed^2 mod n.

for i in range(1, bits):
    # xi:= x[i-1]^2 mod n
    x.append( ( x[i-1]*x[i-1] ) % n)

    # Extração do bit de paridade fazendo um AND
    # lógico e concatenação com o resultado de saída
    # após fazer um deslocamento a esquerda.
    random = (random << 1) | (x[i] & 1)

end = t.time()  # Tempo de término da execução.
total_time = (end - start) * MICRO

print("Random value = " + str(random))
print("%.2f" %total_time + " microssegundos")
\end{lstlisting}

% ----------------------------------------------------------
% Questão 2
% ----------------------------------------------------------
\subsection*{\textbf{2 - Linear congruential generator (LCG):}}
\addcontentsline{toc}{subsection}{2 - Linear congruential generator}

Este gerador é definido por $x_{n} = (aX_{n-1} + c)\mod m$, com $X$ sendo a sequência de valores pseudo-aleatórios, $m$ é o módulo e $m > 0$, $a$ é multiplicador e $0 < a < m$, $c$ o incremento $0\leq c<m$ e $X_0$ é a semente inicial $0 \leq X_0 < m$, $m$ e $c$ precisam ser relativamente primos.

\begin{table}[H]
\centering
\caption{LCG}
\label{lcg-table}
\begin{tabular}{|c|c|}
\hline
\rowcolor[HTML]{C0C0C0} 
\textbf{Tamanho do Número} & \textbf{\begin{tabular}[c]{@{}c@{}}Tempo para gerar\\ (microssegundos)\end{tabular}} \\ \hline
40                         & 37.43                                                                                \\ \hline
56                         & 36.48                                                                                \\ \hline
80                         & 48.88                                                                                \\ \hline
128                        & 103.00                                                                               \\ \hline
168                        & 136.14                                                                               \\ \hline
224                        & 168.09                                                                               \\ \hline
256                        & 175.00                                                                               \\ \hline
512                        & 290.16                                                                               \\ \hline
1024                       & 640.15                                                                               \\ \hline
2048                       & 1334.19                                                                              \\ \hline
4096                       & 2638.82                                                                              \\ \hline
\end{tabular}
\end{table}


\begin{lstlisting}[frame=single, title={lcg.py}]
import sys
import time as t
MICRO = 1000*1000

m = 3141592653589771 # Módulo m
c = 2718281828459051 # Incremento c
a = 2718281828459051 # Multiplicaor a
seed = 2             # Semente X0
random = 0            
x = []                

bits = int(sys.argv[1]) + 1

start = t.time() # Tempo de início da execução.

x.append(seed) # x[0] := seed.

for i in range(1, bits):
    # x[n] = (a * X[n-1] + c) mod m
    x.append( ( a * x[i-1] + c ) % m)
    random = (random << 1) | (x[i] & 1)
    
end = t.time()  # Tempo de término da execução.
total_time = (end - start) * MICRO

print("Random value = " + str(random))
print("%.2f" %total_time + " microssegundos")

\end{lstlisting}

% ----------------------------------------------------------
% Questão 2
% ----------------------------------------------------------
\subsection*{\textbf{3 - Park–Miller random number generator (PM):}}
\addcontentsline{toc}{subsection}{3 - Park–Miller random number generator}

Este gerador é um tipo de LGC e é definido por $X_{i} = g * X_{k-1}\mod n$, onde o módulo $n$ é primo, o multiplicador $g$ é a raíz primitiva de $n$ e a semente $X_0$ é coprima de $n$.

\begin{table}[H]
\centering
\caption{Park-Miller}
\label{pm-table}
\begin{tabular}{|c|c|}
\hline
\rowcolor[HTML]{C0C0C0} 
\textbf{Tamanho do Número} & \textbf{\begin{tabular}[c]{@{}c@{}}Tempo para gerar\\ (microssegundos)\end{tabular}} \\ \hline
40                         & 35.29                                                                                \\ \hline
56                         & 33.62                                                                                \\ \hline
80                         & 45.30                                                                                \\ \hline
128                        & 69.62                                                                                \\ \hline
168                        & 96.08                                                                                \\ \hline
224                        & 122.55                                                                               \\ \hline
256                        & 200.75                                                                               \\ \hline
512                        & 427.48                                                                               \\ \hline
1024                       & 560.76                                                                               \\ \hline
2048                       & 1165.63                                                                              \\ \hline
4096                       & 2506.97                                                                              \\ \hline
\end{tabular}
\end{table}

\begin{lstlisting}[frame=single, title={pm.py}]
import sys
import time as t
MICRO = 1000*1000

n = 3141592653589771 # Módulo n
g = 2718281828459051 # Multiplicador g
seed = 2             # Semente X0
random = 0            
x = []                

bits = int(sys.argv[1]) + 1

start = t.time() # Tempo de início da execução.

x.append(seed) # x[0] := seed.

for i in range(1, bits):
    # X[i] = g * X[k-1] % mod n
    x.append( (g * x[i-1]) % n)
    random = (random << 1) | (x[i] & 1)

end = t.time()  # Tempo de término da execução.
total_time = (end - start) * MICRO

print("Random value = " + str(random))
print("%.2f" %total_time + " microssegundos")
\end{lstlisting}


\section*{\textbf{Respostas}}
\addcontentsline{toc}{subsection}{Conclusão}

Os algoritmos elaborados utilizam uma estrutura muito parecidade, alterando apenas a função \textit{kernel} deles, ou seja sua estrutura é muito similar mudando apenas os cálculos computacionais realizados. Quanto aos testes, existem vários deles, muitos são baterias de testes que visam realizar a verificação da aleatoriedade dos números gerados, utilizando padrões estatísticos. As mais conhecidas destas baterias são o TestU01 e o Diehard.


% ---
% Finaliza a parte no bookmark do PDF, para que se inicie o bookmark na raiz
% ---
\bookmarksetup{startatroot}% 
% ---

% ---
% Conclusão
% ---
% \section*{Considerações finais}
% \addcontentsline{toc}{section}{Considerações finais}

% ----------------------------------------------------------
% ELEMENTOS PÓS-TEXTUAIS
% ----------------------------------------------------------
\postextual

% ----------------------------------------------------------
% Referências bibliográficas
% ----------------------------------------------------------
\nocite{bbs_random}
\nocite{lgc_random}
\nocite{pm_random}
\bibliography{bibliography}

\end{document}
