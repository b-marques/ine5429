%% abtex2-modelo-artigo.tex, v-1.9.6 laurocesar
%% Copyright 2012-2016 by abnTeX2 group at http://www.abntex.net.br/ 
%%
%% This work may be distributed and/or modified under the
%% conditions of the LaTeX Project Public License, either version 1.3
%% of this license or (at your option) any later version.
%% The latest version of this license is in
%%   http://www.latex-project.org/lppl.txt
%% and version 1.3 or later is part of all distributions of LaTeX
%% version 2005/12/01 or later.
%%
%% This work has the LPPL maintenance status `maintained'.
%% 
%% The Current Maintainer of this work is the abnTeX2 team, led
%% by Lauro César Araujo. Further information are available on 
%% http://www.abntex.net.br/
%%
%% This work consists of the files abntex2-modelo-artigo.tex and
%% abntex2-modelo-references.bib
%%

% ------------------------------------------------------------------------
% ------------------------------------------------------------------------
% abnTeX2: Modelo de Artigo Acadêmico em conformidade com
% ABNT NBR 6022:2003: Informação e documentação - Artigo em publicação 
% periódica científica impressa - Apresentação
% ------------------------------------------------------------------------
% ------------------------------------------------------------------------

\documentclass[
    % -- opções da classe memoir --
    article,            % indica que é um artigo acadêmico
    11pt,               % tamanho da fonte
    oneside,            % para impressão apenas no recto. Oposto a twoside
    a4paper,            % tamanho do papel. 
    % -- opções da classe abntex2 --
    %chapter=TITLE,     % títulos de capítulos convertidos em letras maiúsculas
    %section=TITLE,     % títulos de seções convertidos em letras maiúsculas
    %subsection=TITLE,  % títulos de subseções convertidos em letras maiúsculas
    %subsubsection=TITLE % títulos de subsubseções convertidos em letras maiúsculas
    % -- opções do pacote babel --
    english,            % idioma adicional para hifenização
    brazil,             % o último idioma é o principal do documento
    sumario=tradicional,
    ]{abntex2}


% ---
% PACOTES
% ---

% ---
% Pacotes fundamentais 
% ---
\usepackage{lmodern}            % Usa a fonte Latin Modern
\usepackage[T1]{fontenc}        % Selecao de codigos de fonte.
\usepackage[utf8]{inputenc}     % Codificacao do documento (conversão automática dos acentos)
\usepackage{indentfirst}        % Indenta o primeiro parágrafo de cada seção.
\usepackage{nomencl}            % Lista de simbolos
\usepackage{color}              % Controle das cores
\usepackage{graphicx}           % Inclusão de gráficos
\usepackage{microtype}          % para melhorias de justificação
% ---
        
% ---
% Pacotes adicionais, usados apenas no âmbito do Modelo Canônico do abnteX2
% ---
\usepackage{lipsum}             % para geração de dummy text
% ---
        
% ---
% Pacotes de citações
% ---
\usepackage[brazilian,hyperpageref]{backref}     % Paginas com as citações na bibl
\usepackage[alf]{abntex2cite}   % Citações padrão ABNT
% ---

% ---
% Configurações do pacote backref
% Usado sem a opção hyperpageref de backref
\renewcommand{\backrefpagesname}{Citado na(s) página(s):~}
% Texto padrão antes do número das páginas
\renewcommand{\backref}{}
% Define os textos da citação
\renewcommand*{\backrefalt}[4]{
    \ifcase #1 %
        Nenhuma citação no texto.%
    \or
        Citado na página #2.%
    \else
        Citado #1 vezes nas páginas #2.%
    \fi}%
% ---

% ---
% Informações de dados para CAPA e FOLHA DE ROSTO
% ---
\titulo{Relatório INE5429 - Segurança em Computação\\ 
        Pretty Good Privacy (PGP)}
\autor{Bruno Marques do Nascimento\thanks{brunomn95@gmail.com \hspace{1mm} - \hspace{1mm} Universidade Federal de Santa Catarina}}
\instituicao{Universidade Federal de Santa Catarina}
\local{Florianópolis - SC, Brasil}
\data{14 de Março de 2018}
% ---

% ---
% Configurações de aparência do PDF final

% alterando o aspecto da cor azul
\definecolor{blue}{RGB}{41,5,195}

% informações do PDF
\makeatletter
\hypersetup{
        %pagebackref=true,
        pdftitle={\@title}, 
        pdfauthor={\@author},
        pdfsubject={Modelo de artigo científico com abnTeX2},
        pdfcreator={LaTeX with abnTeX2},
        pdfkeywords={abnt}{latex}{abntex}{abntex2}{atigo científico}, 
        colorlinks=true,            % false: boxed links; true: colored links
        linkcolor=blue,             % color of internal links
        citecolor=blue,             % color of links to bibliography
        filecolor=magenta,              % color of file links
        urlcolor=blue,
        bookmarksdepth=4
}
\makeatother
% --- 

% ---
% compila o indice
% ---
\makeindex
% ---

% ---
% Altera as margens padrões
% ---
\setlrmarginsandblock{3cm}{3cm}{*}
\setulmarginsandblock{3cm}{3cm}{*}
\checkandfixthelayout
% ---

% --- 
% Espaçamentos entre linhas e parágrafos 
% --- 

% O tamanho do parágrafo é dado por:
\setlength{\parindent}{1.3cm}

% Controle do espaçamento entre um parágrafo e outro:
\setlength{\parskip}{0.2cm}  % tente também \onelineskip

% Espaçamento simples
\SingleSpacing

% ----
% Início do documento
% ----
\begin{document}

% Seleciona o idioma do documento (conforme pacotes do babel)
%\selectlanguage{english}
\selectlanguage{brazil}

% Retira espaço extra obsoleto entre as frases.
\frenchspacing 

% ----------------------------------------------------------
% ELEMENTOS PRÉ-TEXTUAIS
% ----------------------------------------------------------

%---
%
% Se desejar escrever o artigo em duas colunas, descomente a linha abaixo
% e a linha com o texto ``FIM DE ARTIGO EM DUAS COLUNAS''.
% \twocolumn[           % INICIO DE ARTIGO EM DUAS COLUNAS
%
%---
% página de titulo

\maketitle


% resumo em português
\begin{resumoumacoluna}
    Relatório de aula referente ao primeiro trabalho individual da disciplina INE5429 - Segurança em computação, com o objetivo de conhecer e utilizar o GPG (GNU Privacy Guard) que é uma alternativa GPL do aplicativo PGP de criptografia. Este documento visa responder as perguntas elencadas pelo professor ministrante da disciplina Ricardo Felipe Custódio na plataforma moodle.
 
 \vspace{\onelineskip}
 
\end{resumoumacoluna}

% ]                 % FIM DE ARTIGO EM DUAS COLUNAS
% ---

% ----------------------------------------------------------
% ELEMENTOS TEXTUAIS
% ----------------------------------------------------------
\textual

% ----------------------------------------------------------
% Introdução
% ----------------------------------------------------------
% \section*{Introdução}
% \addcontentsline{toc}{section}{Introdução}


% ----------------------------------------------------------
% Questão 1
% ----------------------------------------------------------
\section{Criar certificado pgp.}


% ----------------------------------------------------------
% Questão 2
% ----------------------------------------------------------
\section{Crie um novo certificado PGP para este trabalho individual (Não use o teu certificado pois ele terá que ser revogado). Coloque esse certificado de testes no servidor PGP. Depois verifique seu status. Então, crie um certificado de revogação e revogue o certificado de testes.}


% ----------------------------------------------------------
% Questão 3
% ----------------------------------------------------------
\section{Pratique a revogação de certificados PGP. Assine um certificado qualquer PGP (de outra pessoa). E envie esse certificado para o servidor PGP. Depois verifique o status do certificado. E então, revogue a assinatura que você fez. Confira o resultado no servidor PGP. Faça um relatório do que você fez, incluindo o KeyID do certificado cuja assinatura você revogou.}


% ----------------------------------------------------------
% Questão 4
% ----------------------------------------------------------
\section{O que é o anel de chaves privadas? Como este está estruturado? Na sua aplicação GPG onde este anel de chaves é armazenado? Quem pode ser acesso a esse porta chaves?}


% ----------------------------------------------------------
% Questão 5
% ----------------------------------------------------------
\section{Qual a diferença entre assinar uma chave local e assinar no servidor?}


% ----------------------------------------------------------
% Questão 6
% ----------------------------------------------------------
\section{O que é e como é organizado o banco de dados de confiabilidade?}


% ----------------------------------------------------------
% Questão 7
% ----------------------------------------------------------
\section{O que são e para que servem as sub-chaves?}


% ----------------------------------------------------------
% Questão 8
% ----------------------------------------------------------
\section{Coloque sua foto (ou uma figura qualquer) que represente você em seu certificado GPG.}


% ----------------------------------------------------------
% Questão 9
% ----------------------------------------------------------
\section{O que é preciso para criar e manter um servidor de chaves GPG, sincronizado com os demais servidores existentes?}


% ----------------------------------------------------------
% Questão 10
% ----------------------------------------------------------
\section{Dê um exemplo de como tornar sigiloso um arquivo usando o GPG. Envie esse arquivo para um colega e que enviar para você outro arquivo cifrado. Você deve decifrar e recuperar o conteúdo original.}


% ----------------------------------------------------------
% Questão 11
% ----------------------------------------------------------
\section{Dê um exemplo de como assinar um arquivo (assinatura anexada e outro com assinatura separada), usando o GPG. Envie uma mensagem assinada para um colega. Esse colega deve enviar para você outra mensagem assinada. Verifique se a assinatura está correta.}

% ---
% Finaliza a parte no bookmark do PDF, para que se inicie o bookmark na raiz
% ---
\bookmarksetup{startatroot}% 
% ---

% ---
% Conclusão
% ---
% \section*{Considerações finais}
% \addcontentsline{toc}{section}{Considerações finais}

% ----------------------------------------------------------
% ELEMENTOS PÓS-TEXTUAIS
% ----------------------------------------------------------
\postextual

% ----------------------------------------------------------
% Referências bibliográficas
% ----------------------------------------------------------
\bibliography{bibliografia}

\end{document}
