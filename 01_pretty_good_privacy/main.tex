%% abtex2-modelo-artigo.tex, v-1.9.6 laurocesar
%% Copyright 2012-2016 by abnTeX2 group at http://www.abntex.net.br/ 
%%
%% This work may be distributed and/or modified under the
%% conditions of the LaTeX Project Public License, either version 1.3
%% of this license or (at your option) any later version.
%% The latest version of this license is in
%%   http://www.latex-project.org/lppl.txt
%% and version 1.3 or later is part of all distributions of LaTeX
%% version 2005/12/01 or later.
%%
%% This work has the LPPL maintenance status `maintained'.
%% 
%% The Current Maintainer of this work is the abnTeX2 team, led
%% by Lauro César Araujo. Further information are available on 
%% http://www.abntex.net.br/
%%
%% This work consists of the files abntex2-modelo-artigo.tex and
%% abntex2-modelo-references.bib
%%

% ------------------------------------------------------------------------
% ------------------------------------------------------------------------
% abnTeX2: Modelo de Artigo Acadêmico em conformidade com
% ABNT NBR 6022:2003: Informação e documentação - Artigo em publicação 
% periódica científica impressa - Apresentação
% ------------------------------------------------------------------------
% ------------------------------------------------------------------------

\documentclass[
    % -- opções da classe memoir --
    article,            % indica que é um artigo acadêmico
    11pt,               % tamanho da fonte
    oneside,            % para impressão apenas no recto. Oposto a twoside
    a4paper,            % tamanho do papel. 
    % -- opções da classe abntex2 --
    %chapter=TITLE,     % títulos de capítulos convertidos em letras maiúsculas
    %section=TITLE,     % títulos de seções convertidos em letras maiúsculas
    %subsection=TITLE,  % títulos de subseções convertidos em letras maiúsculas
    %subsubsection=TITLE % títulos de subsubseções convertidos em letras maiúsculas
    % -- opções do pacote babel --
    english,            % idioma adicional para hifenização
    brazil,             % o último idioma é o principal do documento
    sumario=tradicional,
    ]{abntex2}


% ---
% PACOTES
% ---

% ---
% Pacotes fundamentais 
% ---
\usepackage{lmodern}            % Usa a fonte Latin Modern
\usepackage[T1]{fontenc}        % Selecao de codigos de fonte.
\usepackage[utf8]{inputenc}     % Codificacao do documento (conversão automática dos acentos)
% \usepackage{indentfirst}        % Indenta o primeiro parágrafo de cada seção.
\usepackage{nomencl}            % Lista de simbolos
\usepackage{color}              % Controle das cores
\usepackage{graphicx}           % Inclusão de gráficos
\usepackage{microtype}          % para melhorias de justificação
% ---
        
% ---
% Pacotes adicionais, usados apenas no âmbito do Modelo Canônico do abnteX2
% ---
\usepackage{lipsum}             % para geração de dummy text
\usepackage{fancyvrb}
\usepackage{todonotes}
% ---
        
% ---
% Pacotes de citações
% ---
\usepackage[brazilian,hyperpageref]{backref}     % Paginas com as citações na bibl
\usepackage[alf]{abntex2cite}   % Citações padrão ABNT
% ---

% ---
% Configurações do pacote backref
% Usado sem a opção hyperpageref de backref
\renewcommand{\backrefpagesname}{Citado na(s) página(s):~}
% Texto padrão antes do número das páginas
\renewcommand{\backref}{}
% Define os textos da citação
\renewcommand*{\backrefalt}[4]{
    \ifcase #1 %
        Nenhuma citação no texto.%
    \or
        Citado na página #2.%
    \else
        Citado #1 vezes nas páginas #2.%
    \fi}%
% ---

% ---
% Informações de dados para CAPA e FOLHA DE ROSTO
% ---
\titulo{Relatório INE5429 - Segurança em Computação\\ 
        Pretty Good Privacy (PGP)}
\autor{Bruno Marques do Nascimento\thanks{brunomn95@gmail.com \hspace{1mm} - \hspace{1mm} Universidade Federal de Santa Catarina}}
\instituicao{Universidade Federal de Santa Catarina}
\local{Florianópolis - SC, Brasil}
\data{14 de Março de 2018}
% ---

% ---
% Configurações de aparência do PDF final

% alterando o aspecto da cor azul
\definecolor{blue}{RGB}{41,5,195}

% informações do PDF
\makeatletter
\hypersetup{
        %pagebackref=true,
        pdftitle={\@title}, 
        pdfauthor={\@author},
        pdfsubject={Modelo de artigo científico com abnTeX2},
        pdfcreator={LaTeX with abnTeX2},
        pdfkeywords={abnt}{latex}{abntex}{abntex2}{atigo científico}, 
        colorlinks=true,            % false: boxed links; true: colored links
        linkcolor=blue,             % color of internal links
        citecolor=blue,             % color of links to bibliography
        filecolor=magenta,              % color of file links
        urlcolor=blue,
        bookmarksdepth=4
}
\makeatother
% --- 

% ---
% compila o indice
% ---
\makeindex
% ---

% ---
% Altera as margens padrões
% ---
\setlrmarginsandblock{3cm}{3cm}{*}
\setulmarginsandblock{3cm}{3cm}{*}
\checkandfixthelayout
% ---

% --- 
% Espaçamentos entre linhas e parágrafos 
% --- 

% O tamanho do parágrafo é dado por:
\setlength{\parindent}{1.3cm}

% Controle do espaçamento entre um parágrafo e outro:
\setlength{\parskip}{0.2cm}  % tente também \onelineskip

% Espaçamento simples
\SingleSpacing

% ----
% Início do documento
% ----
\begin{document}

% Seleciona o idioma do documento (conforme pacotes do babel)
%\selectlanguage{english}
\selectlanguage{brazil}

% Retira espaço extra obsoleto entre as frases.
\frenchspacing 

% ----------------------------------------------------------
% ELEMENTOS PRÉ-TEXTUAIS
% ----------------------------------------------------------

%---
%
% Se desejar escrever o artigo em duas colunas, descomente a linha abaixo
% e a linha com o texto ``FIM DE ARTIGO EM DUAS COLUNAS''.
% \twocolumn[           % INICIO DE ARTIGO EM DUAS COLUNAS
%
%---
% página de titulo

\maketitle


% resumo em português
\begin{resumoumacoluna}
    Relatório de aula referente ao primeiro trabalho individual da disciplina INE5429 - Segurança em computação, com o objetivo de conhecer e utilizar o GPG (GNU Privacy Guard) que é uma alternativa GPL do aplicativo PGP de criptografia. Este documento visa responder as perguntas elencadas pelo professor ministrante da disciplina Ricardo Felipe Custódio na plataforma moodle.
 
 \vspace{\onelineskip}
 
\end{resumoumacoluna}

% ]                 % FIM DE ARTIGO EM DUAS COLUNAS
% ---

% ----------------------------------------------------------
% ELEMENTOS TEXTUAIS
% ----------------------------------------------------------
\textual

% ----------------------------------------------------------
% Introdução
% ----------------------------------------------------------
% \section*{Introdução}
% \addcontentsline{toc}{section}{Introdução}

\section*{\textbf{Questões e respostas:}}
\addcontentsline{toc}{section}{Questões e respostas:}
% ----------------------------------------------------------
% Questão 1
% ----------------------------------------------------------
\subsection*{\textbf{1 - Criar certificado pgp.}}
\addcontentsline{toc}{subsection}{Questão 1}

Através do terminal utilizado o comando \texttt{gpg ----full--gen--key}, o passo a passo abaixo foi realizado e o seguinte certificado gerado:

\begin{Verbatim}[frame=single, commandchars=\\\{\}, fontsize=\footnotesize]
Kind of key:  RSA and RSA.
Keysize:      2048 bits.
Valid for:    180 days.
Key expires:  dom 09 set 2018 10:55:33 -03.
USER-ID:      Bruno Marques (ine5429) <brunomn95@gmail.com>
KEY-ID:       48B783E542BC3DE0
    
pub   rsa2048 2018-03-13 [SC] [expires: 2018-09-09]
      FCF80DBA20F4AF28F329261448B783E542BC3DE0
uid   Bruno Marques (ine5429) <brunomn95@gmail.com>
sub   rsa2048 2018-03-13 [E] [expires: 2018-09-09]
\end{Verbatim}

% ----------------------------------------------------------
% Questão 2
% ----------------------------------------------------------
\subsection*{\textbf{2 - Crie um novo certificado PGP para este trabalho individual (Não use o teu certificado pois ele terá que ser revogado). Coloque esse certificado de testes no servidor PGP. Depois verifique seu status. Então, crie um certificado de revogação e revogue o certificado de testes.}}
\addcontentsline{toc}{subsection}{Questão 2}

Um novo certificado para posterior revogação foi gerado com os atributos abaixo elencados:

\begin{Verbatim}[frame=single, commandchars=\\\{\}, fontsize=\footnotesize]
Kind of key:  RSA and RSA.
Keysize:      2048 bits.
Valid for:    180 days.
Key expires:  dom 09 set 2018 11:51:51 -03
USER-ID:      Bruno Marques (questao2-ine5429) <brunomn@gmail.com>
KEY-ID:       6E9ECD78A9CC5EBB

pub   rsa2048 2018-03-13 [SC] [expires: 2018-09-09]
      D5E5F581B6BCB811AAF4300A6E9ECD78A9CC5EBB
uid   Bruno Marques (questao2-ine5429) <brunomn@gmail.com>
sub   rsa2048 2018-03-13 [E] [expires: 2018-09-09]
\end{Verbatim}

Feito isto, o mesmo foi colocado no servidor PGP através do comando: \texttt{gpg ----keyserver keyserver.cais.rnp.br ----send-keys 6E9ECD78A9CC5EBB}. Gerando o seguinte output:

\begin{Verbatim}[frame=single, commandchars=\\\{\}, fontsize=\footnotesize]
gpg: sending key 6E9ECD78A9CC5EBB to hkp://keyserver.cais.rnp.br
\end{Verbatim}

Após enviar o certificado ao servidor foi verificado seu status atraveś do comando \texttt{gpg ----keyserver keyserver.cais.rnp.br ----search--keys 6E9ECD78A9CC5EBB}. Que gerou a saída abaixo:

\begin{Verbatim}[frame=single, commandchars=\\\{\}, fontsize=\footnotesize]
gpg: data source: http://keyserver.cais.rnp.br:11371
(1) Bruno Marques (questao2-ine5429) <brunomn@gmail.com>
    2048 bit RSA key 6E9ECD78A9CC5EBB, created: 2018-03-13, expires: 2018-09-09
Keys 1-1 of 1 for "6E9ECD78A9CC5EBB".
\end{Verbatim}

Em seguida foi criado um certificado de revogação através do comando: \texttt{gpg ----gen--revoke "Bruno Marques (questao2--ine5429)"}. Que gerou o seguinte certificado de revogação:

\begin{Verbatim}[frame=single, commandchars=\\\{\}, fontsize=\footnotesize]
ASCII armored output forced.
-----BEGIN PGP PUBLIC KEY BLOCK-----
Comment: This is a revocation certificate

iQEfBCABCAAJBQJaqAenAh0AAAoJEG6ezXipzF67e8kIALk9q9kybAmfokksqhQK
3bQXg3QAqszZkYZe8HqLsk/Ca5KLUfoZ/LzfXQ5vHHehaFOr1M+9ZiNrlpGZncwt
cMOsQJyUZyhLf5/V46AiGNG6xAEiQ7Nsxpbbi4Uq+jnf5IakqWY3pTii/PIihlFa
rKgya0Kw7NhOzJYfKu0pZ9A8g2e4TjK+xWf4Xk96FPnKmbOZcjf/vcQ/qHTcX4Hg
AdrP0xhRyMgBpxvfAOVjUanGy5rxoz1vllne4pyHx9sFJP7tjS9ICa9F86Ww8ik5
iNxBgbOkfGNA+K84ldNaDc1PjjIKbChINh/CZ3x2INBGJ4xuvW8OrQI4qM6LctLr
npk=
=fCkI
-----END PGP PUBLIC KEY BLOCK-----
Revocation certificate created.
\end{Verbatim}

Este certificado de revogação foi então salvo no arquivo ``revog.txt'' e utilizou-se o comando \texttt{gpg ----import  \textasciitilde /.gnupg/openpgp--revocs.d/revog.txt} para realizar a revogação desta chave e que posteriormente foi enviada ao servidor PGP, através do comando \texttt{gpg ----keyserver keyserver.cais.rnp.br ----send--keys 6E9ECD78A9CC5EBB}. Para verificar seu status utilizou-se novamento o comando \texttt{gpg ----keyserver keyserver.cais.rnp.br ----search--keys 6E9ECD78A9CC5EBB} que mostrou o certificado como revogado:

\begin{Verbatim}[frame=single, commandchars=\\\{\}, fontsize=\footnotesize]
gpg: data source: http://keyserver.cais.rnp.br:11371
(1) Bruno Marques (questao2-ine5429) <brunomn@gmail.com>
    2048 bit RSA key 6E9ECD78A9CC5EBB, created: 2018-03-13, expires: 2018-09-09 \textbf{(revoked)}
\end{Verbatim}



% ----------------------------------------------------------
% Questão 3
% ----------------------------------------------------------
\subsection*{\textbf{3 - Pratique a revogação de certificados PGP. Assine um certificado qualquer PGP (de outra pessoa). E envie esse certificado para o servidor PGP. Depois verifique o status do certificado. E então, revogue a assinatura que você fez. Confira o resultado no servidor PGP. Faça um relatório do que você fez, incluindo o KeyID do certificado cuja assinatura você revogou.}}
\addcontentsline{toc}{subsection}{Questão 3}

Foi importado um certificado para a máquina local através do comando \texttt{gpg ----recv--keys 54f950865065a318}. E então utilizados os comandos \texttt{gpg ----sign--key 54f950865065a318} e \texttt{gpg ----keyserver keyserver.cais.rnp.br ----send--keys 54f950865065a318} para respectivamente assiná-lo e enviá-lo para o servidor pgp. Para verificar os status foi utilizado o seguinte comando \texttt{gpg ----list--sigs 54f950865065a318} que resultou no output abaixo:

\begin{Verbatim}[frame=single, commandchars=\\\{\}, fontsize=\footnotesize]
pub   rsa4096 2014-09-22 [SCA] [expires: 2019-09-21]
      E805BAF5D4688A9ED6658B8C54F950865065A318
uid           [  full  ] Jean Martina <jean.martina@ufsc.br>
sig          FD3D4401F5343946 2015-03-19  [User ID not found]
sig          BF510D9287C98027 2015-03-20  [User ID not found]
sig          1CFC45B2CB3BCBEE 2015-03-20  [User ID not found]
sig          5DF66B47F48274D9 2015-03-27  [User ID not found]
sig          A40097C5149F62B3 2015-03-27  [User ID not found]
sig          A7E4BB4DDEA7DDF0 2015-04-16  [User ID not found]
sig 1        3AD79D3F318B7BA5 2017-08-16  [User ID not found]
sig        1 D84870658D9AA66A 2017-01-24  [User ID not found]
rev          3AD79D3F318B7BA5 2017-08-16  [User ID not found]
sig          C47D8F38399DE079 2015-03-20  [User ID not found]
sig          C7BC91E9273DD888 2015-03-21  [User ID not found]
sig 3        0F0F3FF23A2E6AB6 2015-04-15  [User ID not found]
sig 3        54F950865065A318 2014-09-22  Jean Martina <jean.martina@ufsc.br>
sig 3        54F950865065A318 2015-02-20  Jean Martina <jean.martina@ufsc.br>
sig          48B783E542BC3DE0 2018-03-14  Bruno Marques (ine5429) <brunomn95@gmail.com>
\end{Verbatim}

Para revogar a assinatura foi utilizado o comando \texttt{gpg ----edit--key 54f950865065a318} seguido de \texttt{revsig} para criar o certificado de revogação de assinatura. E por fim foi utilizado o comando \texttt{save} para salvar as edições na chave, que fora enviada novamente ao servidor \texttt{gpg ----keyserver keyserver.cais.rnp.br ----send--keys 54f950865065a318}. Em seguida verificou-se novamente seu status:

\begin{Verbatim}[frame=single, commandchars=\\\{\}, fontsize=\footnotesize]
pub   rsa4096 2014-09-22 [SCA] [expires: 2019-09-21]
      E805BAF5D4688A9ED6658B8C54F950865065A318
uid           [ unknown] Jean Martina <jean.martina@ufsc.br>
\textbf{rev          48B783E542BC3DE0 2018-03-14  Bruno Marques (ine5429) <brunomn95@gmail.com>}
sig          FD3D4401F5343946 2015-03-19  [User ID not found]
sig          BF510D9287C98027 2015-03-20  [User ID not found]
sig          1CFC45B2CB3BCBEE 2015-03-20  [User ID not found]
sig          5DF66B47F48274D9 2015-03-27  [User ID not found]
sig          A40097C5149F62B3 2015-03-27  [User ID not found]
sig          A7E4BB4DDEA7DDF0 2015-04-16  [User ID not found]
sig 1        3AD79D3F318B7BA5 2017-08-16  [User ID not found]
sig        1 D84870658D9AA66A 2017-01-24  [User ID not found]
rev          3AD79D3F318B7BA5 2017-08-16  [User ID not found]
sig          C47D8F38399DE079 2015-03-20  [User ID not found]
sig          C7BC91E9273DD888 2015-03-21  [User ID not found]
sig 3        0F0F3FF23A2E6AB6 2015-04-15  [User ID not found]
sig 3        54F950865065A318 2014-09-22  Jean Martina <jean.martina@ufsc.br>
sig 3        54F950865065A318 2015-02-20  Jean Martina <jean.martina@ufsc.br>
sig          48B783E542BC3DE0 2018-03-14  Bruno Marques (ine5429) <brunomn95@gmail.com>
\end{Verbatim}



% ----------------------------------------------------------
% Questão 4
% ----------------------------------------------------------
\subsection*{\textbf{4 - O que é o anel de chaves privadas? Como este está estruturado? Na sua aplicação GPG onde este anel de chaves é armazenado? Quem pode ser acesso a esse porta chaves?}}
\addcontentsline{toc}{subsection}{Questão 4}

O anel de chaves privadas é uma estrutura de dados responsável em armazenar o par de chaves pública/privada que pertecem a um determinado dono, ou seja, esse anel encontra-se presente apenas na máquina local do respectivo dono da chave. Logo, quem possuir acesso a conta deste usuário na máquina terá acesso ao anel de chaves privadas, porém quando uma pessoa acessa o anel de chaves privadas para utilizar a chave privada, ela deverá forncer a senha que apenas o dono da chave privada deve possuir.
O anel de chaves privadas pode ser visualizado como uma tabela de banco de dados, onde cada linha representa um par de chave pública/privada pertencente ao usuário, além disso, essa tabela contém os seguintes atributos \cite{Stallings:2010:CNS:1824151}: 
\begin{Verbatim}[commandchars=\\\{\}, fontsize=\footnotesize]
    • Timestamp:    O timestamp de quando o par de chaves foi gerado.
    • Key ID:       Os 64bits menos significativos da chave pública do par.
    • Public key:   A chave pública do par.
    • Private key:  A chave privada do par, CIFRADA.
    • User ID:      Identificador do usuário.
\end{Verbatim}


% ----------------------------------------------------------
% Questão 5
% ----------------------------------------------------------
\subsection*{\textbf{5 - Qual a diferença entre assinar uma chave local e assinar no servidor?}}
\addcontentsline{toc}{subsection}{Questão 5}

A diferença é que ao assinar localmente, a chave só terá essa assinatura na máquina local e outros usuários não poderão visualizar a sua assinatura sobre esta chave, quando esta assinatura está no servidor, todos os usuários podem acessar o servidor e verificar as assinaturas que uma determinada chave recebeu.

% ----------------------------------------------------------
% Questão 6
% ----------------------------------------------------------
\subsection*{\textbf{6 - O que é e como é organizado o banco de dados de confiabilidade?}}
\addcontentsline{toc}{subsection}{Questão 6}

O banco de dados de confiabilidade, também conhecido como anel de chaves públicas, pode ser visto também como uma tabela onde cada linha representa um certificado de chave pública. Associado a cada chava pública existe um campo de ``legitimidade da chave'' que indica o nível de confiança da PGP sobre esta chave pública e sua validade perante determinado usuário. Este campo de legitimidade é influenciado diretamente pelas assinaturas que um determinado certificado recebeu e consequentemente pelas assinaturas que as assinaturas possuem e assim sucessivamente. É desta maneira que a confiabilidade é calculada e estabelecida. 


% ----------------------------------------------------------
% Questão 7
% ----------------------------------------------------------
\subsection*{\textbf{7 - O que são e para que servem as sub-chaves?}}
\addcontentsline{toc}{subsection}{Questão 7}

Sub-chaves são como chaves normais, porém elas estão automaticamente relacionadas a um par de chaves principal. Elas servem tanto para assinatura como paraa cifragem. Apesar de estarem realcionadas a um par de chaves principal elas podem ser armazenadas separadamente do par de chave principal, assim como serem revogadas independentemente.

% ----------------------------------------------------------
% Questão 8
% ----------------------------------------------------------
\subsection*{\textbf{8 - Coloque sua foto (ou uma figura qualquer) que represente você em seu certificado GPG.}}
\addcontentsline{toc}{subsection}{Questão 8}

Os seguintes comandos foram utilizados para isto: 
\begin{Verbatim}[commandchars=\\\{\}, fontsize=\footnotesize]
    \texttt{gpg --edit-key 42BC3DE0}
    \texttt{addphoto}
    \texttt{save}
    \texttt{gpg --keyserver keyserver.cais.rnp.br --send-keys 42BC3DE0}
\end{Verbatim}


% ----------------------------------------------------------
% Questão 9
% ----------------------------------------------------------
\subsection*{\textbf{9 - O que é preciso para criar e manter um servidor de chaves GPG, sincronizado com os demais servidores existentes?}}
\addcontentsline{toc}{subsection}{Questão 9}

Basicamente é necessário ter um servidor web que ficará conectado continuamente na internet, importar uma base de dados de chaves públicas já existente para não precisar popular uma base de dados do zero, e instalar e executar um daemon que ficará responsável em manter o banco de dados do seu servidor sincronizado com os demais. Além disso, para permitir que os usuário interajam via web com o servidor de chaves é necessário configurar um site para isto.

% ----------------------------------------------------------
% Questão 10
% ----------------------------------------------------------
\subsection*{\textbf{10 - Dê um exemplo de como tornar sigiloso um arquivo usando o GPG. Envie esse arquivo para um colega e que enviar para você outro arquivo cifrado. Você deve decifrar e recuperar o conteúdo original.}}
\addcontentsline{toc}{subsection}{Questão 10}

Um exemplo de tornar um arquivo sigiloso é mostrado a seguir. Primeiramente é necessário ter a chave pública do destinatário localmente e para isso utilizou-se o comando  \texttt{gpg ----recv--key B461AD97}, feito isso cria-se o arquivo, por exemplo, ``sigiloso.txt'', que será cifrado para se tornar de fato sigiloso, para isso utiliza-se o seguinte comando \texttt{gpg ----encrypt ----recipient B461AD97 sigiloso.txt} que irá gerar o arquivo cifrado ``sigiloso.txt.gpg'' que será enviado ao destinatário por e-mail (\texttt{johannwestphall@gmail.com}).

O arquivo que recebi tem o nome de ``RelatorioSeg.gpg'', para decifrar este arquivo e recuperar seu conteúdo original o seguinte comando foi utilizado \texttt{gpg ----decrypt RelatorioSeg.gpg} que revelou o conteúdo do arquivo, que consistia das questões em branco deste relatório. 

% ----------------------------------------------------------
% Questão 11
% ----------------------------------------------------------
\subsection*{\textbf{11 - Dê um exemplo de como assinar um arquivo (assinatura anexada e outro com assinatura separada), usando o GPG. Envie uma mensagem assinada para um colega. Esse colega deve enviar para você outra mensagem assinada. Verifique se a assinatura está correta.}}
\addcontentsline{toc}{subsection}{Questão 11}

Para assinar um arquivo com assinatura anexada foi utilizado o seguinte comando \texttt{gpg --s attach--sign.txt} que gerou o arquivo ``attach--sign.txt.gpg'' apartir do qual é possível fazer a verificação da assinatura do arquivo pelo comando \texttt{gpg ----verify attach--sign.txt.gpg}. Para assinatura separada o seguinte comando foi utilizado \texttt{gpg --b detach--sign.txt} que gerou o arquivo ``detach--sign.txt.sig'', para realizar a verificação da assinatura do documento é necessário utilizar o documento em texto puro e o arquivo assinado com o seguinte comando \texttt{gpg ----verify detach--sign.txt.sig detach--sign.txt}.

O arquivo recebido para verificação de assinatura foi o ``RelatorioSegAssinado.gpg'', a verificação de assinatura foi feita com o seguinte comando \texttt{gpg ----verify "RelatorioSeg
Assinado.gpg"} tendo como saída o texto abaixo.

\begin{Verbatim}[frame=single, commandchars=\\\{\}, fontsize=\footnotesize]
gpg: Signature made qua 14 mar 2018 18:00:57 -03
gpg:                using RSA key 6AE4C628B461AD97
gpg: Good signature from "teste (teste) <johannwestphall@gmail.com>"
\end{Verbatim}


% ---
% Finaliza a parte no bookmark do PDF, para que se inicie o bookmark na raiz
% ---
\bookmarksetup{startatroot}% 
% ---

% ---
% Conclusão
% ---
% \section*{Considerações finais}
% \addcontentsline{toc}{section}{Considerações finais}

% ----------------------------------------------------------
% ELEMENTOS PÓS-TEXTUAIS
% ----------------------------------------------------------
\postextual

% ----------------------------------------------------------
% Referências bibliográficas
% ----------------------------------------------------------
\nocite{GnuPrivacyGuardHowto}
\nocite{subkeys_debian}
\bibliography{bibliography}

\end{document}
