%% abtex2-modelo-artigo.tex, v-1.9.6 laurocesar
%% Copyright 2012-2016 by abnTeX2 group at http://www.abntex.net.br/ 
%%
%% This work may be distributed and/or modified under the
%% conditions of the LaTeX Project Public License, either version 1.3
%% of this license or (at your option) any later version.
%% The latest version of this license is in
%%   http://www.latex-project.org/lppl.txt
%% and version 1.3 or later is part of all distributions of LaTeX
%% version 2005/12/01 or later.
%%
%% This work has the LPPL maintenance status `maintained'.
%% 
%% The Current Maintainer of this work is the abnTeX2 team, led
%% by Lauro César Araujo. Further information are available on 
%% http://www.abntex.net.br/
%%
%% This work consists of the files abntex2-modelo-artigo.tex and
%% abntex2-modelo-references.bib
%%

% ------------------------------------------------------------------------
% ------------------------------------------------------------------------
% abnTeX2: Modelo de Artigo Acadêmico em conformidade com
% ABNT NBR 6022:2003: Informação e documentação - Artigo em publicação 
% periódica científica impressa - Apresentação
% ------------------------------------------------------------------------
% ------------------------------------------------------------------------

\documentclass[
    % -- opções da classe memoir --
    article,            % indica que é um artigo acadêmico
    11pt,               % tamanho da fonte
    oneside,            % para impressão apenas no recto. Oposto a twoside
    a4paper,            % tamanho do papel. 
    % -- opções da classe abntex2 --
    %chapter=TITLE,     % títulos de capítulos convertidos em letras maiúsculas
    %section=TITLE,     % títulos de seções convertidos em letras maiúsculas
    %subsection=TITLE,  % títulos de subseções convertidos em letras maiúsculas
    %subsubsection=TITLE % títulos de subsubseções convertidos em letras maiúsculas
    % -- opções do pacote babel --
    english,            % idioma adicional para hifenização
    brazil,             % o último idioma é o principal do documento
    sumario=tradicional,
    ]{abntex2}


% ---
% PACOTES
% ---

% ---
% Pacotes fundamentais 
% ---
\usepackage{lmodern}            % Usa a fonte Latin Modern
\usepackage[T1]{fontenc}        % Selecao de codigos de fonte.
\usepackage[utf8]{inputenc}     % Codificacao do documento (conversão automática dos acentos)
% \usepackage{indentfirst}        % Indenta o primeiro parágrafo de cada seção.
\usepackage{nomencl}            % Lista de simbolos
\usepackage{color}              % Controle das cores
\usepackage{graphicx}           % Inclusão de gráficos
\usepackage{microtype}          % para melhorias de justificação
% ---
        
% ---
% Pacotes adicionais, usados apenas no âmbito do Modelo Canônico do abnteX2
% ---
\usepackage{lipsum}             % para geração de dummy text
\usepackage{fancyvrb}
\usepackage{todonotes}
% ---
        
% ---
% Pacotes de citações
% ---
\usepackage[brazilian,hyperpageref]{backref}     % Paginas com as citações na bibl
\usepackage[alf]{abntex2cite}   % Citações padrão ABNT
% ---

% ---
% Configurações do pacote backref
% Usado sem a opção hyperpageref de backref
\renewcommand{\backrefpagesname}{Citado na(s) página(s):~}
% Texto padrão antes do número das páginas
\renewcommand{\backref}{}
% Define os textos da citação
\renewcommand*{\backrefalt}[4]{
    \ifcase #1 %
        Nenhuma citação no texto.%
    \or
        Citado na página #2.%
    \else
        Citado #1 vezes nas páginas #2.%
    \fi}%
% ---

% ---
% Informações de dados para CAPA e FOLHA DE ROSTO
% ---
\titulo{INE5429 - Segurança em Computação\\ 
        Trabalho Individual 01 - Capitulo 03}
\autor{Bruno Marques do Nascimento\thanks{brunomn95@gmail.com \hspace{1mm} - \hspace{1mm} Universidade Federal de Santa Catarina}}
\instituicao{Universidade Federal de Santa Catarina}
\local{Florianópolis - SC, Brasil}
\data{14 de Março de 2018}
% ---

% ---
% Configurações de aparência do PDF final

% alterando o aspecto da cor azul
\definecolor{blue}{RGB}{41,5,195}

% informações do PDF
\makeatletter
\hypersetup{
        %pagebackref=true,
        pdftitle={\@title}, 
        pdfauthor={\@author},
        pdfsubject={Modelo de artigo científico com abnTeX2},
        pdfcreator={LaTeX with abnTeX2},
        pdfkeywords={abnt}{latex}{abntex}{abntex2}{atigo científico}, 
        colorlinks=true,            % false: boxed links; true: colored links
        linkcolor=blue,             % color of internal links
        citecolor=blue,             % color of links to bibliography
        filecolor=magenta,              % color of file links
        urlcolor=blue,
        bookmarksdepth=4
}
\makeatother
% --- 

% ---
% compila o indice
% ---
\makeindex
% ---

% ---
% Altera as margens padrões
% ---
\setlrmarginsandblock{3cm}{3cm}{*}
\setulmarginsandblock{3cm}{3cm}{*}
\checkandfixthelayout
% ---

% --- 
% Espaçamentos entre linhas e parágrafos 
% --- 

% O tamanho do parágrafo é dado por:
\setlength{\parindent}{1.3cm}

% Controle do espaçamento entre um parágrafo e outro:
\setlength{\parskip}{0.2cm}  % tente também \onelineskip

% Espaçamento simples
\SingleSpacing

% ----
% Início do documento
% ----
\begin{document}

% Seleciona o idioma do documento (conforme pacotes do babel)
%\selectlanguage{english}
\selectlanguage{brazil}

% Retira espaço extra obsoleto entre as frases.
\frenchspacing 

% ----------------------------------------------------------
% ELEMENTOS PRÉ-TEXTUAIS
% ----------------------------------------------------------

%---
%
% Se desejar escrever o artigo em duas colunas, descomente a linha abaixo
% e a linha com o texto ``FIM DE ARTIGO EM DUAS COLUNAS''.
% \twocolumn[           % INICIO DE ARTIGO EM DUAS COLUNAS
%
%---
% página de titulo

\maketitle


% resumo em português
\begin{resumoumacoluna}
    Este documento é referente ao primeiro trabalho individual da disciplina INE5429 - Segurança em computação, com o objetivo de responder algumas questões do capítulo 3 do livro texto da disciplina. Este documento visa responder as perguntas elencadas pelo professor ministrante da disciplina Ricardo Felipe Custódio na plataforma moodle.
 
 \vspace{\onelineskip}
 
\end{resumoumacoluna}

% ]                 % FIM DE ARTIGO EM DUAS COLUNAS
% ---

% ----------------------------------------------------------
% ELEMENTOS TEXTUAIS
% ----------------------------------------------------------
\textual

% ----------------------------------------------------------
% Introdução
% ----------------------------------------------------------
% \section*{Introdução}
% \addcontentsline{toc}{section}{Introdução}

\section*{\textbf{Questões e respostas:}}
\addcontentsline{toc}{section}{Questões e respostas:}


% ----------------------------------------------------------
% Questão 1
% ----------------------------------------------------------
\subsection*{\textbf{3.5 3.9 - Consider the substitution defined by row 1 of S-box S\textsubscript{1} in Table S.2. Show a block diagram similar to Figure 3.2 that corresponds to this substitution.}}
\addcontentsline{toc}{subsection}{Questão 3.9}


% ----------------------------------------------------------
% Questão 2
% ----------------------------------------------------------
\subsection*{\textbf{3.6 3.10 - Compute the bits number 1, 16, 33, and 48 at the output of the first round of the DES decryption, assuming that the ciphertext block is composed of all ones and the external key is composed of all ones.}}
\addcontentsline{toc}{subsection}{Questão 3.10}


% ----------------------------------------------------------
% Questão 3
% ----------------------------------------------------------
\subsection*{\textbf{3.8 3.11 - This problem provides a numerical example of encryption using a one-round version of DES. We start with the same bit pattern for the key K and the plaintext, namely:}}
\addcontentsline{toc}{subsection}{Questão 3.11}
\begin{Verbatim}
Hexadecimal notation:  0 1 2 3 4 5 6 7 8 9 A B C D E F

Binary notation:       0000 0001 0010 0011 0100 0101 0110 0111
                       1000 1001 1010 1011 1100 1101 1110 1111
\end{Verbatim}
\subsubsection*{a. Derive K\textsubscript{1}, the frst-round subkey.}
\subsubsection*{b. Derive L\textsubscript{0}, R\textsubscript{0}.}
\subsubsection*{c. Expand R\textsubscript{0} to get E[R\textsubscript{0}], where E[\#] is the expansion function of Table S.1.}
\subsubsection*{d. Calculate A = E[R\textsubscript{0}] $\oplus$ K1.}
\subsubsection*{e. Group the 48-bit result of (d) into sets of 6 bits and evaluate the corresponding S-box substitutions.}
\subsubsection*{f. Concatenate the results of (e) to get a 32-bit result, B.}
\subsubsection*{g. Apply the permutation to get P(B).}
\subsubsection*{h. Calculate R\textsubscript{1} = P(B) $\oplus$ L\textsubscript{0}.}
\subsubsection*{i. Write down the ciphertext.}


% ----------------------------------------------------------
% Questão 4
% ----------------------------------------------------------
\subsection*{\textbf{3.11 3.12 - Compare the initial permutation table (Table S.1a) with the permuted choice one table (Table S.3b). Are the structures similar? If so, describe the similarities. What conclusions can you draw from this analysis?}}
\addcontentsline{toc}{subsection}{Questão 3.12}


% ----------------------------------------------------------
% Questão 5
% ----------------------------------------------------------
\subsection*{\textbf{3.16 - Refer to Figure G.2, which depicts key generation for S-DES.}}
\addcontentsline{toc}{subsection}{Questão 3.16}
\subsubsection*{a. How important is the initial P10 permutation function?}
\subsubsection*{b. How important are the two LS-1 shift functions?}

% ----------------------------------------------------------
% Questão 6
% ----------------------------------------------------------
\subsection*{\textbf{3.17 - The equations for the variables q and r for S-DES are defined in the section on S-DES analysis. Provide the equations for s and t.}}
\addcontentsline{toc}{subsection}{Questão 3.17}


% ----------------------------------------------------------
% Questão 7
% ----------------------------------------------------------
\subsection*{\textbf{3.18 - Using S-DES, decrypt the string (10100010) using the key (0111111101) by hand. Show intermediate results after each function (IP,F\textsubscript{K},SW,F\textsubscript{K},IP\textsuperscript{-1}). Then decode the first 4 bits of the plaintext string to a letter and the second 4 bits to another letter where we encode A through P in base 2 (i.e., A = 0000, B = 0001, ..., P = 1111). \textit{Hint}: As a midway check, after the application of SW, the string should be (00010011).}}
\addcontentsline{toc}{subsection}{Questão 3.18}



% ---
% Finaliza a parte no bookmark do PDF, para que se inicie o bookmark na raiz
% ---
\bookmarksetup{startatroot}% 
% ---

% ---
% Conclusão
% ---
% \section*{Considerações finais}
% \addcontentsline{toc}{section}{Considerações finais}

% ----------------------------------------------------------
% ELEMENTOS PÓS-TEXTUAIS
% ----------------------------------------------------------
\postextual

% ----------------------------------------------------------
% Referências bibliográficas
% ----------------------------------------------------------
\nocite{Stallings:2005:CNS:1076613}
\nocite{Stallings:2010:CNS:1824151}
\bibliography{bibliography}

\end{document}