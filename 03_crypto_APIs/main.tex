%% abtex2-modelo-artigo.tex, v-1.9.6 laurocesar
%% Copyright 2012-2016 by abnTeX2 group at http://www.abntex.net.br/ 
%%
%% This work may be distributed and/or modified under the
%% conditions of the LaTeX Project Public License, either version 1.3
%% of this license or (at your option) any later version.
%% The latest version of this license is in
%%   http://www.latex-project.org/lppl.txt
%% and version 1.3 or later is part of all distributions of LaTeX
%% version 2005/12/01 or later.
%%
%% This work has the LPPL maintenance status `maintained'.
%% 
%% The Current Maintainer of this work is the abnTeX2 team, led
%% by Lauro César Araujo. Further information are available on 
%% http://www.abntex.net.br/
%%
%% This work consists of the files abntex2-modelo-artigo.tex and
%% abntex2-modelo-references.bib
%%

% ------------------------------------------------------------------------
% ------------------------------------------------------------------------
% abnTeX2: Modelo de Artigo Acadêmico em conformidade com
% ABNT NBR 6022:2003: Informação e documentação - Artigo em publicação 
% periódica científica impressa - Apresentação
% ------------------------------------------------------------------------
% ------------------------------------------------------------------------

\documentclass[
    % -- opções da classe memoir --
    article,            % indica que é um artigo acadêmico
    11pt,               % tamanho da fonte
    oneside,            % para impressão apenas no recto. Oposto a twoside
    a4paper,            % tamanho do papel. 
    % -- opções da classe abntex2 --
    %chapter=TITLE,     % títulos de capítulos convertidos em letras maiúsculas
    %section=TITLE,     % títulos de seções convertidos em letras maiúsculas
    %subsection=TITLE,  % títulos de subseções convertidos em letras maiúsculas
    %subsubsection=TITLE % títulos de subsubseções convertidos em letras maiúsculas
    % -- opções do pacote babel --
    english,            % idioma adicional para hifenização
    brazil,             % o último idioma é o principal do documento
    sumario=tradicional,
    ]{abntex2}


% ---
% PACOTES
% ---

% ---
% Pacotes fundamentais 
% ---
\usepackage{lmodern}            % Usa a fonte Latin Modern
\usepackage[T1]{fontenc}        % Selecao de codigos de fonte.
\usepackage[utf8]{inputenc}     % Codificacao do documento (conversão automática dos acentos)
% \usepackage{indentfirst}        % Indenta o primeiro parágrafo de cada seção.
\usepackage{nomencl}            % Lista de simbolos
\usepackage{color}              % Controle das cores
\usepackage{graphicx}           % Inclusão de gráficos
\usepackage{microtype}          % para melhorias de justificação
% ---
        
% ---
% Pacotes adicionais, usados apenas no âmbito do Modelo Canônico do abnteX2
% ---
\usepackage{lipsum}             % para geração de dummy text
\usepackage{fancyvrb}
\usepackage{todonotes}
\usepackage{listings}
% ---
        
% ---
% Pacotes de citações
% ---
\usepackage[brazilian,hyperpageref]{backref}     % Paginas com as citações na bibl
\usepackage[alf]{abntex2cite}   % Citações padrão ABNT
% ---

% ---
% Configurações do pacote backref
% Usado sem a opção hyperpageref de backref
\renewcommand{\backrefpagesname}{Citado na(s) página(s):~}
% Texto padrão antes do número das páginas
\renewcommand{\backref}{}
% Define os textos da citação
\renewcommand*{\backrefalt}[4]{
    \ifcase #1 %
        Nenhuma citação no texto.%
    \or
        Citado na página #2.%
    \else
        Citado #1 vezes nas páginas #2.%
    \fi}%
% ---

% ---
% Informações de dados para CAPA e FOLHA DE ROSTO
% ---
\titulo{Relatório INE5429 - Segurança em Computação\\ 
        Provedores de Serviços Criptográficos}
\autor{Bruno Marques do Nascimento\thanks{brunomn95@gmail.com \hspace{1mm} - \hspace{1mm} Universidade Federal de Santa Catarina}}
\instituicao{Universidade Federal de Santa Catarina}
\local{Florianópolis - SC, Brasil}
\data{01 de Abril de 2018}
% ---

% ---
% Configurações de aparência do PDF final

% alterando o aspecto da cor azul
\definecolor{blue}{RGB}{41,5,195}

% informações do PDF
\makeatletter
\hypersetup{
        %pagebackref=true,
        pdftitle={\@title}, 
        pdfauthor={\@author},
        pdfsubject={Modelo de artigo científico com abnTeX2},
        pdfcreator={LaTeX with abnTeX2},
        pdfkeywords={abnt}{latex}{abntex}{abntex2}{atigo científico}, 
        colorlinks=true,            % false: boxed links; true: colored links
        linkcolor=blue,             % color of internal links
        citecolor=blue,             % color of links to bibliography
        filecolor=magenta,              % color of file links
        urlcolor=blue,
        bookmarksdepth=4
}
\makeatother
% --- 

% ---
% compila o indice
% ---
\makeindex
% ---

% ---
% Altera as margens padrões
% ---
\setlrmarginsandblock{3cm}{3cm}{*}
\setulmarginsandblock{3cm}{3cm}{*}
\checkandfixthelayout
% ---

% --- 
% Espaçamentos entre linhas e parágrafos 
% --- 

% O tamanho do parágrafo é dado por:
\setlength{\parindent}{1.3cm}

% Controle do espaçamento entre um parágrafo e outro:
\setlength{\parskip}{0.2cm}  % tente também \onelineskip

% Espaçamento simples
\SingleSpacing

% ----
% Início do documento
% ----
\begin{document}

% Seleciona o idioma do documento (conforme pacotes do babel)
%\selectlanguage{english}
\selectlanguage{brazil}

% Retira espaço extra obsoleto entre as frases.
\frenchspacing 

% ----------------------------------------------------------
% ELEMENTOS PRÉ-TEXTUAIS
% ----------------------------------------------------------

%---
%
% Se desejar escrever o artigo em duas colunas, descomente a linha abaixo
% e a linha com o texto ``FIM DE ARTIGO EM DUAS COLUNAS''.
% \twocolumn[           % INICIO DE ARTIGO EM DUAS COLUNAS
%
%---
% página de titulo

\maketitle


% resumo em português
\begin{resumoumacoluna}
    Relatório de aula da disciplina INE5429 - Segurança em computação, com o objetivo de conhecer e comparar provedores de serviços criptográficos. Este documento visa responder as perguntas elencadas pelo professor ministrante da disciplina Ricardo Felipe Custódio na plataforma moodle.
 
 \vspace{\onelineskip}
 
\end{resumoumacoluna}

% ]                 % FIM DE ARTIGO EM DUAS COLUNAS
% ---

% ----------------------------------------------------------
% ELEMENTOS TEXTUAIS
% ----------------------------------------------------------
\textual

% ----------------------------------------------------------
% Introdução
% ----------------------------------------------------------
% \section*{Introdução}
% \addcontentsline{toc}{section}{Introdução}

\section*{\textbf{Comparação entre Crypto++ e cryptlib.}}
\addcontentsline{toc}{section}{Questões e respostas:}

\subsection*{\textbf{1 - Geral:}}
\addcontentsline{toc}{subsection}{Geral}

\par\noindent
\begin{minipage}[t]{.49\textwidth}
\begin{Verbatim}[frame=single]
             Crypto ++

    Crypto++ é uma biblioteca grá-
tis e de código aberto para a lin-
guagem de programação C++. Lançada
no ano de 1995, esta biblioteca su-
porta tanto arquitetura 32s como 64
bits, nas mais diversas plataformas
e sistemas operacionais. É uma bi-
blioteca de alto nível, onde usuá-
rios não necessitam entender a fun-
do como os algoritmos criptográfi-
cos estão sendo implementados.
\end{Verbatim}
\end{minipage}
\hfill
\begin{minipage}[t]{.49\textwidth}
\begin{Verbatim}[frame=single]
             Cryptlib

    Também é uma biblioteca de có-
digo aberto, esta escrita em C. Ela
prove uma interface de alto nível,
permitindo ao usuário adicionar re-
cursos de segurança e criptografia
sem a necessidade de saber detalhes
em baixo nível dos algoritmos crip-
tográficos de autenticação e cifra-
gem/decifragem.


\end{Verbatim}
\end{minipage}

\subsection*{\textbf{2 - Arquitetura:}}
\addcontentsline{toc}{subsection}{Arquiteturas}

\par\noindent
\begin{minipage}[t]{.49\textwidth}
\begin{Verbatim}[frame=single]
             Crypto ++

    Esta é uma biblioteca que for-
nece implementações criptográficas,
diferente da cryptlib, seu nível de
abstração é um pouco menor exigindo
que o usuário conheça bem o funcio-
namento dos algorítmos criptográfi-
cos que pretende utilizar. Por ser
open source sua arquitetura permite
que o usuário também implemente al-
goritmos próprios e os acople à bi-
blioteca.


\end{Verbatim}
\end{minipage}
\hfill
\begin{minipage}[t]{.49\textwidth}
\begin{Verbatim}[frame=single]
             Cryptlib

    Consiste de um conjunto de ca-
madas de serviços de segurança, in-
do de interfaces de alto nível até 
camadas mais baixas onde os algorí-
timos utilizados estão implementa-
dos. Essas várias camadas na arqui-
tetura permite que os mais diversos 
níveis de usuários utilizem a bi-
blioteca, dando a liberdade para 
usuários experientes poderem escre-
ver os próprios algorítimos para 
trabalhar em conjunto com a biblio-
teca.
\end{Verbatim}
\end{minipage}

\subsection*{\textbf{3 - Algoritmos assimétricos:}}
\addcontentsline{toc}{subsection}{Algoritmos assimétricos}

\par\noindent
\begin{minipage}[t]{.49\textwidth}
\begin{Verbatim}[frame=single]
             Crypto ++

• RSA
• DSA
• Determinsitic DSA (RFC 6979)
• ElGamal
• Nyberg-Rueppel (NR)
• Rabin-Williams (RW)
• ECGDSA
• LUC
• LUCELG
• DLIES (variants of DHAES)
• ESIGN

\end{Verbatim}
\end{minipage}
\hfill
\begin{minipage}[t]{.49\textwidth}
\begin{Verbatim}[frame=single]
             Cryptlib

• RSA
• DSA

• ElGamal


• ECDSA





\end{Verbatim}
\end{minipage}


\subsection*{\textbf{4 - Algoritmos simétricos:}}
\addcontentsline{toc}{subsection}{Algoritmos simétricos}
\par\noindent
\begin{minipage}[t]{.49\textwidth}
\begin{Verbatim}[frame=single]
             Crypto ++

• AES
• ARIA
• ARC4
• Blowfish
• BTEA
• Camellia
• CAST128
• CAST256
• DES
• 2-key Triple-DES
• 3-key Triple-DES
• DESX
• GOST
• IDEA
• Luby-Rackoff
• Kalyna (128/256/512)
• MARS
• RC2
• RC5
• RC6
• SAFER-K
• SAFER-SK
• SEED
• Serpent
• SHACAL-2
• SHARK
• SKIPJACK
• SM4
• Square
• TEA
• 3-Way
• Threefish (256/512/1024)
• Twofish
• XTEA
• ChaCha (ChaCha-8/12/20)
• Panama-LE
• Panama-BE
• Salsa20
• SEAL-LE
• SEAL-BE
• WAKE
• XSalsa20
\end{Verbatim}
\end{minipage}
\hfill
\begin{minipage}[t]{.49\textwidth}
\begin{Verbatim}[frame=single]
             Cryptlib

• AES

• RC4 (ARC4)




• CAST

• DES
• Triple-DES


• IDEA




























\end{Verbatim}
\end{minipage}

\subsection*{\textbf{5 - Funções hash:}}
\addcontentsline{toc}{subsection}{Funções hash}

\par\noindent
\begin{minipage}[t]{.49\textwidth}
\begin{Verbatim}[frame=single]
             Crypto ++

• BLAKE2b
• BLAKE2s
• Keccack (F1600)
• SHA-1
• SHA-224
• SHA-256
• SHA-384
• SHA-512
• SHA-3
• Poly1305
• SipHash
• Tiger
• RIPEMD (128, 256, 160, 320)
• SM3
• WHIRLPOOL
• MD2
• MD4
• MD5
\end{Verbatim}
\end{minipage}
\hfill
\begin{minipage}[t]{.49\textwidth}
\begin{Verbatim}[frame=single]
             Cryptlib




• SHA-1
• SHA-2
• SHA-256


• SHA-3



• RIPMED-160

• WHIRLPOOL


• MD5
\end{Verbatim}
\end{minipage}

\subsection*{\textbf{6 - Funções mac:}}
\addcontentsline{toc}{subsection}{Funções mac}

\par\noindent
\begin{minipage}[t]{.49\textwidth}
\begin{Verbatim}[frame=single]
             Crypto ++

• BLAKE2b
• BLAKE2s
• CMAC
• CBC-MAC
• DMAC
• GMAC (GCM)
• HMAC
• Poly1305
• SipHash
• Two-Track-MAC
• VMAC
\end{Verbatim}
\end{minipage}
\hfill
\begin{minipage}[t]{.49\textwidth}
\begin{Verbatim}[frame=single]
             Cryptlib







• HMAC-SHA1
• HMAC-SHA2 MAC



\end{Verbatim}
\end{minipage}

\subsection*{\textbf{7 - Suporte a hardware:}}
\addcontentsline{toc}{subsection}{Suporte a hardware}

\par\noindent
\begin{minipage}[t]{.49\textwidth}
\begin{Verbatim}[frame=single]
             Crypto ++











\end{Verbatim}
\end{minipage}
\hfill
\begin{minipage}[t]{.49\textwidth}
\begin{Verbatim}[frame=single]
             Cryptlib

• Crypto smart cards
• Dallas iButtons
• Datakeys/iKeys
• Fortezza cards
• Hardware crypto accelerators
• Hardware security modules (HSMs)
• PCMCIA crypto tokens
• PCI crypto cards
• PKCS #11 devices
• USB tokens
\end{Verbatim}
\end{minipage}


\subsection*{\textbf{8 - Algoritmos de geração:}}
\addcontentsline{toc}{subsection}{Algoritmos de geração}

\par\noindent
\begin{minipage}[t]{.49\textwidth}
\begin{Verbatim}[frame=single]
             Crypto ++

• Key generation/derivation
• Prime number generation
• Pseudo random number generation:
  - NullRNG()
  - LC_RNG
  - RandomPool
  - BlockingRng
  - NonblockingRng
  - AutoSeededRandomPool
  - AutoSeededX917RNG
  - NIST Hash_DRBG and HMAC_DRBG
  - RDRAND
  - RDSEED
  - MersenneTwister (MT19937 and
                     MT19937-AR)
\end{Verbatim}
\end{minipage}
\hfill
\begin{minipage}[t]{.49\textwidth}
\begin{Verbatim}[frame=single]
             Cryptlib

• Key generation/derivation














\end{Verbatim}
\end{minipage}

\subsection*{\textbf{9 - Algoritmos de acordo de chaves:}}
\addcontentsline{toc}{subsection}{Algoritmos de acordo de chaves}

\par\noindent
\begin{minipage}[t]{.49\textwidth}
\begin{Verbatim}[frame=single]
             Crypto ++

• Diffie-Hellman (DH)
• Unified Diffie-Hellman (DH2)
• Menezes-Qu-Vanstone (MQV)
• Hashed MQV (HMQV)
• Fully Hashed MQV (FHMQV)
• ECDH
• ECMQV
• ECHMQV
• ECFHMQV
• LUCDIF
• XTR-DH
\end{Verbatim}
\end{minipage}
\hfill
\begin{minipage}[t]{.49\textwidth}
\begin{Verbatim}[frame=single]
             Cryptlib

• Diffie-Hellman (DH)




• ECDH





\end{Verbatim}
\end{minipage}


\subsection*{\textbf{10 - Tipo de licença:}}
\addcontentsline{toc}{subsection}{Tipo de licença}

\par\noindent
\begin{minipage}[t]{.49\textwidth}
\begin{Verbatim}[frame=single]
             Crypto ++

    A versão compilada está sob di-
reitos autorais de acordo com a li-
cença "Boost Software License 1.0".
Enquanto que os arquivos individu-
ais na compilação são todos de do-
mínio público.
\end{Verbatim}
\end{minipage}
\hfill
\begin{minipage}[t]{.49\textwidth}
\begin{Verbatim}[frame=single]
             Cryptlib

    É distribuido sob licença dupla
que permite uso livre e open-source
sob uma licença compatível com a 
GPL denominada Sleepycat. E para
código fechado o uso sobre uma li-
cença comercial padrão.
\end{Verbatim}
\end{minipage}

 \vspace{\onelineskip}
 \vspace{\onelineskip}
 \vspace{\onelineskip}

% ---
% Finaliza a parte no bookmark do PDF, para que se inicie o bookmark na raiz
% ---
\bookmarksetup{startatroot}% 
% ---

% ---
% Conclusão
% ---
% \section*{Considerações finais}
% \addcontentsline{toc}{section}{Considerações finais}

% ----------------------------------------------------------
% ELEMENTOS PÓS-TEXTUAIS
% ----------------------------------------------------------
\postextual

% ----------------------------------------------------------
% Referências bibliográficas
% ----------------------------------------------------------
\nocite{cryptlib}
\nocite{cryptlib_manual}
\nocite{cryptopp}
\bibliography{bibliography}

\end{document}
