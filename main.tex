% revision history
%
% 20140901 NPS: initial version

\documentclass[11pt]{article}
\usepackage{fancyhdr}
\usepackage{url}
\topmargin=-5mm
\evensidemargin=0cm
\oddsidemargin=0cm
\textwidth=16cm
\textheight=22cm
\addtolength{\headheight}{1.6pt}
\newcommand{\cancel}[1]{}
\newcommand{\mymark}{$^*$}

\newcommand{\lastupdate}{Sept 2014}

\lhead{\sc IACR Guidelines for Authors}
\rhead{\sc \lastupdate}

\title{\bf IACR Guidelines for Authors}
\author{\mbox{}}

\date{\lastupdate
 \footnote{The most recent version of this document
    can be obtained from \protect\url{http://www.iacr.org/docs/}.\newline
  Editors of this document: Nigel Smart (2014).}}


\begin{document}

\pagestyle{fancy}
\pagenumbering{arabic}

\maketitle

\noindent Dear Author,
\vspace*{4mm}
\par\noindent
The purpose of this document is to clarify aspects of submission 
to a publication of the IACR.
%
These guidelines complement other guidelines available on the IACR
website, in particular, the Guidelines for Program Chairs, the IACR
Policy on Irregular Submissions, and the Guidelines for Reviewers.


\paragraph{Confidentiality.}
All submissions to IACR conferences and workshops are in confidence.
Reviewers are not allowed to disclose information about the authors, the
content of submissions, other reviews, or discussions in a PC to
anyone else not in such a role role.
Likewise you may not ask reviewers and PC members for information
about your submission \emph{before} the PC decisions are made; nor may
you ask for information about discussions after the event (except, perhaps,
for asking on general advice about a re-submission).
In all cases any questions must be addressed to the program chairs
rather than to individual PC members.

\paragraph{Conflicts of interest.}
Your papers will not be reviewed by reviewers who have a \emph{conflict of
  interest} with at least one author of the submission.  The IACR does
not impose a detailed policy on conflicts of interest.  The editor or
Program Chair decides on what constitutes a conflict according to high
standards in terms of scientific integrity~--- at least colleagues
from the same research group, people in a current or very recent
student-advisor relationship, close friends, and family members have a
conflict.

\paragraph{Attendance.}
By submitting to an IACR conference or workshop the authors are
committing to present their paper if it is accepted. Authors should
make themselves aware before submitting of any visa application
times or other restrictions before submitting.
Only in exceptional circumstances will a replacement speaker
be accepted by the program chair. 
Repeat offenders will be reported to the IACR Ethics Committee.

\paragraph{Anonymity.}
For IACR conferences (Asiacrypt, Crypto and Eurocrypt)
all submissions are anonymous. 
No author names appear on a submission, and no funding information
or identifying information should appear within the document.
It is however acceptable to post full versions of your work
on the Cryptology ePrint Achive, give presentations of your work
etc.
For IACR workshops, each workshop has its own policy
on anonymous submissions. At the time of writing (2014) only TCC allows,
and mandates, non-anonymous submissions.

\paragraph{Cryptology ePrint Archive.}
The IACR \emph{encourages} authors to place all submissions on
the Cryptology ePrint Archive as soon as practically possible.
This does not conflict with the anonymity requirement above, as
anonymity is there to protect referees who do not wish to 
know who an author is.
Placing your article on the Cryptology ePrint Archive enables a
public priority date to be set.
Authors should be aware that multiple versions can be made
available of a paper via the update functionality; with the 
latest revision appearing as the ``main'' one. 

For accepted papers we encourage authors to upload their camera ready 
version to the Cryptology ePrint Archive as well; and authors who fail to do
so may find the IACR loads up a version on their behalf.

\paragraph{Irregular submissions.}
You should be aware of the IACR policy on irregular submissions.
Irregular submissions typically fall in two categories:
\begin{itemize}
\item \emph{Parallel submissions:} A parallel submission occurs when
  authors submit essentially the same material to one or more other
  publication venues with overlapping reviewing periods.
\item \emph{Plagiarism:} Plagiarism arises when substantial parts of
  existing publications are copied and submitted, virtually unchanged,
  without the addition of new material, and without proper attribution
  of the source, by other ``author(s).''
\end{itemize}
Such submissions will be rejected when detected; and a report will be made
to the IACR Ethics Committee. 
Authors should be aware that most security conferences share information
in relation to parallel submission, and that plagarism is now easily
detected using online tools.
Action may be taken against authors who conduct such unethical behavior.

The IACR recognizes that some work may not fit into a standard
conference format. We therefore encourage, where it makes sense,
for authors to submit two ``related'' works to a conference or workshop.
If this is done the author should explicitly contact the program
chair about this submmission, as special referee assignments may be
needed.
The authors should be aware that the two works should be able to
stand alone, and the outcome may be two, one or none of the papers
are accepted.

\paragraph{Sticky Reviews.}
IACR acknowledges that the process of submitting a rejected paper
from one venue to another can lead to disparity of reviewing
opinions and to additional workload for reviewers. Thus IACR
\emph{encourages} authors to include in their Supplementary 
Material\footnote{Sometimes called ``The Appendix''} responses
to reviews from \emph{previous IACR events}. We would like to
extend this to non-IACR events, but this would require the 
permission of the PC chair of the prior non-IACR event.

Note, that the referee's of the new paper will not have access to the
old version, or the referee's reports, thus your comments should be
understandable without these items.
Including comments to say you have addressed a referee comment helps
if a referee who is seeing your paper for the second time, by enabling
them to concentrate on whether you have made the changes suggested.

Such a response should be in the following form:
\begin{verbatim}
This paper was previously submitted to XXXXX 2014, YYYY 2013.

The paper has been revised according to the feedback from the referees. 
We state and address the relevant referee's comments below.

Referee said : The paper is incremental
Response     : We disagree with the referee because they did not see .....

Referee said : There is a typo on page 3,....
Response     : We thank the referee for spotting this, and have corrected this.

blah blah
\end{verbatim}
We encourage all authors to be respectful to prior referees' opinions;
just as we encourage referees to respect the work of authors.



\end{document}
